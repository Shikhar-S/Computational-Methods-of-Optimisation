\documentclass[a4paper, 12pt, fleqn]{article}

\usepackage{amsmath}
\usepackage{amsthm}
\usepackage{amssymb}
\usepackage{xcolor}
\usepackage{float}
\usepackage{hyperref}
\usepackage{url}
\usepackage[margin=1in]{geometry}

\hypersetup{
    colorlinks,
    linkcolor={red!50!black},
    citecolor={red!50!black},
    urlcolor={blue!50!black}
}

\newtheorem{theorem}{Theorem}[section]
\newtheorem{definition}{Definition}[section]
\newtheorem{example}{Example}[section]
\newtheorem{corollary}{Corollary}[theorem]
\newtheorem{lemma}[theorem]{Lemma}

\newcommand*{\floor}[1]{\left\lfloor #1 \right\rfloor}
\newcommand*{\ceil}[1]{\left\lceil #1 \right\rceil}

\newcommand{\initFromContents}{
\tableofcontents
\newpage
\setlength{\parindent}{0pt}
\setlength{\parskip}{0.5em}
}

\newcommand{\initAfterBeginDocument}{
\maketitle
\initFromContents{}
}

\newcommand{\addMyBib}{
\bibliographystyle{plainurl}
\bibliography{bibdb}
}

\date{}

\input{src/cmo/common.texlib}

\title{CMO: Goldstein and Wolfe optimization}

\begin{document}

\maketitle
\initMinimal{}

\begin{definition}
$C^1_L$ is the subset of $C^1$ functions for which
\[ \| \grad_f(x) - \grad_f(z) \| \le L\|x-z\| \]
This is called the Lipschitz condition.
\end{definition}

\textbf{Objective}: Minimize a lower-bounded $C^1_L$ function $f: \mathbb{R}^d \mapsto \mathbb{R}$.

\tableofcontents

\section{Goldstein and Wolfe conditions}

Let $u$ be a direction of decrease at $x^{(i)}$ (i.e. $\grad_f(x^{(i)})^Tu < 0$).
Our descent algorithm will repeatedly choose a direction of descent
(not necessarily steepest descent) and move in that direction with magnitude $\alpha$.

Unlike the previous algorithms we saw, we'll not necessarily pick $\alpha$
as $\operatorname{argmin}_{\alpha > 0} f(x + \alpha u)$.
This is called \textbf{inexact line search}.
But this doesn't mean we can pick $\alpha$ arbitrarily.
We still have to be smart about picking $\alpha$ to guarantee (quick) convergence.
There are 2 famous ways of picking $\alpha$: by the Goldstein conditions
and the Wolfe conditions.

Let $g(\alpha) = f(x^{(i)} + \alpha u)$.
Therefore, $g'(0) = \grad_f(x^{(i)})^Tu < 0$.
Also, $g$ is lower bounded because $f$ is lower-bounded.

Draw a line which passes through $(0, g(0))$ with slope $m_1 g'(0)$, where $0 < m_1 < 1$
(note that the slope is negative).
Let $h_1(\alpha) = g(0) + m_1 g'(0) \alpha$ be that line.
Let $t(\alpha) = h_1(\alpha) - g(\alpha)$.

\begin{lemma}
$t$ has a positive zero. Let $\overline{\alpha}_1$ be the smallest positive zero.
Then $t$ is positive in the interval $(0, \overline{\alpha}_1)$. Formally,
\[ \exists \overline{\alpha}_1 > 0, (t(\overline{\alpha}_1) = 0 \wedge
(\forall \alpha \in (0, \overline{\alpha}_1), t(\alpha) > 0)) \]
\end{lemma}
\begin{proof}
Let $f^*$ be the minimum value of $f$.
\[ h_1(\alpha) - g(\alpha) < 0 \Leftarrow h_1(\alpha) - f^* < 0 \iff \alpha > \frac{f^* - g(0)}{m_1g'(0)} > 0 \]
Therefore, there is an $\alpha$ for which $t(\alpha) < 0$.

Since $g \in C^1$, by Taylor series, we get that for very small positive $\alpha$,
\[ g(\alpha) = g(0) + g'(0)\alpha + \alpha o(1) \]
\[ \implies t(\alpha) = \alpha( (1-m_1)(-g'(0)) + o(1) ) > 0 \]
Therefore, there is an $\alpha$ for which $t(\alpha) > 0$.

Since $g$ is continuous, by the intermediate value theorem,
there must be an $\overline{\alpha}_1 > 0$ for which $t(\overline{\alpha}_1) = 0$.
Without loss of generality, assume that $\overline{\alpha}_1$ is the smallest positive zero of $t$.
Since $t(\alpha) > 0$ for small positive $\alpha$,
$t(\alpha) > 0$ for all $\alpha \in (0, \overline{\alpha}_1)$.
\end{proof}

In our descent algorithm, if we choose $\alpha$ from the interval $(0, \overline{\alpha}_1)$,
then $g(0) = h_1(0) > h_1(\alpha) > g(\alpha)$. This means that
$f(x^{(i)}) > f(x^{(i)} + \alpha u)$, which is what we required.

However, the decrease may be too small, especially if $\alpha$ is very close to 0.
To counteract this, we'll impose another condition on $\alpha$.
We have 2 choices here.

\subsection{Goldstein condition}

Let $h_2(\alpha) = g(0) + m_2 g'(0) \alpha$, where $0 < m_1 < m_2 < 1$.
Therefore, $h_2 - g$ has a smallest positive zero $\overline{\alpha}_2$.
Also, $\overline{\alpha}_2 < \overline{\alpha}_1$.
We'll chose $\alpha$ from the interval $(\overline{\alpha}_2, \overline{\alpha_1})$.
This is called the Goldstein condition for choosing $\alpha$.

\subsection{Wolfe condition}

Choose an $\alpha \in (0, \overline{\alpha}_1)$
such that $g'(\alpha) \ge m_3g'(0)$, where $m_3 \in (0, 1)$.
This is called the Wolfe condition.

\begin{theorem}
If $m_3 \ge m_1$, it's possible to satisfy the Wolfe condition.
\end{theorem}
\begin{proof}
Suppose we choose $\widehat{\alpha} \in (0, \overline{\alpha}_1)$.
Since $g$ is differentiable, by mean value theorem, we get
\[ \exists \alpha \in [\widehat{\alpha}, \overline{\alpha}_1],
    g'(\alpha)(\overline{\alpha}_1 - \widehat{\alpha}) = g(\overline{\alpha}_1) - g(\widehat{\alpha}) \]
Combine the above result with $g(\widehat{\alpha}) < h_1(\widehat{\alpha})$
and $g(\overline{\alpha}) = h_1(\overline{\alpha})$ to get $g'(\alpha) > g'(0)m_1$.

If we choose $m_3 \ge m_1$, then $g'(0)m_3 \le g'(0)m_1 < g'(\alpha)$.
Therefore, the Wolfe condition is satisfied for some
$\alpha \in (0, \overline{\alpha}_1)$.
\end{proof}

\section{Convergence of Wolfe condition}

\begin{align*}
& g'(\alpha) \ge m_3g'(0) \tag{by Wolfe condition}
\\ &\Rightarrow \grad_f(x^{(i)} + \alpha u)^T u \ge m_3 \grad_f(x^{(i)})^T u
\\ &\Rightarrow \left(\grad_f(x^{(i)} + \alpha u) - \grad_f(x^{(i)})\right)^T u \ge -(1-m_3) \grad_f(x^{(i)})^T u
    \tag{subtract $\grad_f(x^{(i)})^T u$ from both sides}
\\ &\Rightarrow \left\| \grad_f(x^{(i)} + \alpha u) - \grad_f(x^{(i)}) \right\| \ge -(1-m_3) \grad_f(x^{(i)})^T u
    \tag{both sides were +ve. Apply Cauchy-Schwarz inequality}
\\ &\Rightarrow L\alpha\|u\|^2 \ge -(1-m_3) \grad_f(x^{(i)})^T u
    \tag{Lipschitz condition}
\\ &\Rightarrow \alpha \ge \frac{-(1-m_3)\grad_f(x^{(i)})^T u}{L\|u\|^2}
\end{align*}

\begin{align*}
& g(\alpha) < h_1(\alpha) = g(0) + m_1 g'(0) \alpha
\\ &\Rightarrow f(x^{(i+1)}) < f(x^{(i)}) + m_1 \grad_f(x^{(i)})^T u \alpha
\\ &\Rightarrow f(x^{(i)}) - f(x^{(i+1)})
    > \frac{m_1(1-m_3)}{L} \left(\frac{\grad_f(x^{(i)})^Tu}{\|u\|}\right)^2
\end{align*}

Let $\grad_f(x^{(i)})^T u = -\cos\theta_i \left\|\grad_f(x^{(i)})\right\|\|u\|$.
We'll impose another constraint: we'll choose $u$ to not just be the descent direction,
but also in a way that $\cos\theta_i$ is lower-bounded by a positive constant.
\[ f(x^{(i)}) - f(x^{(i+1)}) \ge \frac{m_1(1-m_3)}{L} \cos^2\theta_i \|\grad_f(x^{(i)})\|^2 \]
Summing $i$ from $0$ to $T-1$, we get
\[ \forall T, f(x^{(i)}) - f^* \ge f(x^{(0)}) - f(x^{(T)})
\ge \frac{m_1(1-m_3)}{L} \sum_{i=0}^{T-1} \cos^2\theta_i \left\| \grad_f(x^{(i)}) \right\|^2 \]

$\therefore \sum_{i=0}^{\infty} \cos^2\theta_i \left\| \grad_f(x^{(i)}) \right\|^2$
is a convergent series. So for $i \rightarrow \infty$, $\grad_f(x^{(i)}) \rightarrow 0$.

Therefore, for $i \rightarrow \infty$, $x^{(i)}$ approaches a stationary point.
Therefore, the descent algorithm which uses Wolfe condition converges to a stationary point,
which would hopefully be a local minimum.

\section{Alternate Characterization of \texorpdfstring{$C^1_L$}{C1L}}

Let $f \in C^1_L$. Let $g(\alpha) = f(x + \alpha(y - x))$.
Then $g'(\alpha) = \grad_f(x + \alpha(y-x))^T(y-x)$.
Therefore, $g(0) = f(x)$, $g(1) = f(y)$ and $g'(0) = \grad_f(x)^T(y-x)$.
\[ \int_0^1 (g'(\alpha) - g'(0))d\alpha = f(y) - f(x) - \grad_f(x)^T(y-x) \]

\begin{align*}
& \left| f(y) - f(x) - \grad_f(x)^T(y-x) \right|
\\ &= \left| \int_0^1 (g'(\alpha) - g'(0)) d\alpha \right|
\\ &\le \int_0^1 |g'(\alpha) - g'(0)| d\alpha
\\ &= \int_0^1 \left| \left( \grad_f(x + \alpha(y-x)) - \grad_f(x) \right)^T(y-x) \right| d\alpha
\\ &\le \int_0^1 \left\| \grad_f(x + \alpha(y-x)) - \grad_f(x) \right\| \|y-x\| d\alpha
    \tag{Cauchy-Schwarz inequality}
\\ &\le \int_0^1 L\alpha \|y-x\|^2 d\alpha  \tag{Lipschitz condition}
\\ &= \frac{L}{2} \|y-x\|^2
\end{align*}

\section{Convergence of Goldstein condition}

Let $u = \grad_f(x^{(i)})$ and $x^{(i+1)} = x^{(i)} - \alpha u$.

Let $g(\alpha) = f(x^{(i)} - \alpha u)$.
Then $g'(0) = - \grad_f(x^{(i)})^T u = -\|u\|^2$.

$h_1(\alpha) = g(0) + m_1 g'(0) \alpha = f(x^{(i)}) - \alpha m_1 \|u\|^2$.
Similarly $h_2(\alpha) = f(x^{(i)}) - \alpha m_2 \|u\|^2$.

\begin{align*}
& h_2(\alpha) \le g(\alpha) \le h_1(\alpha)
\\ &\Rightarrow f(x^{(i)}) - m_2 \alpha \|u\|^2 \le f(x^{(i+1)}) \le f(x^{(i)}) - m_1 \alpha \|u\|^2
\\ &\Rightarrow m_1 \alpha \|u\|^2 \le f(x^{(i)}) - f(x^{(i+1)}) \le m_2 \alpha \|u\|^2
\end{align*}

\begin{align*}
& f(x^{(i)}) - f(x^{(i+1)}) + \grad_f(x^{(i)})^T(x^{(i+1)} - x^{(i)})
\\ &\le m_2 \alpha \|u\|^2 + \grad_f(x^{(i)})^T(x^{(i+1)} - x^{(i)})
\\ &= m_2 \alpha \|u\|^2 - \alpha \|u\|^2
\\ &= -(1-m_2) \alpha \|u\|^2
\end{align*}

Therefore, by Lipschitz condition,
\begin{align*}
& (1-m_2) \alpha \|u\|^2 \le \frac{L}{2} \|x^{(i+1)} - x^{(i)}\|^2 = \frac{L\alpha^2\|u\|^2}{2}
\\ &\implies \alpha \ge \frac{2(1-m_2)}{L}
\\ &\implies \frac{2(1-m_2)m_1}{L}\|u\|^2 \le m_1 \alpha \|u\|^2 \le f(x^{(i)}) - f(x^{(i+1)})
\\ &\implies \forall T, \frac{2(1-m_2)m_1}{L} \sum_{i=0}^{T-1} \left\|\grad_f(x^{(i)})\right\|^2
    \le f(x^{(0)}) - f(x^{(T)}) \le f(x^{(0)}) - f^*
\\ &\implies \forall T, \sum_{i=0}^{T-1} \left\|\grad_f(x^{(i)})\right\|^2
    \le \frac{(f(x^{(0)}) - f^*)L}{2m_1(1-m_2)}
\end{align*}

$\therefore \sum_{i=0}^{\infty} \left\| \grad_f(x^{(i)}) \right\|^2$
is a convergent series. So for $i \rightarrow \infty$, $\grad_f(x^{(i)}) \rightarrow 0$.

Therefore, for $i \rightarrow \infty$, $x^{(i)}$ approaches a stationary point.
Therefore, the descent algorithm which uses Goldstein condition converges to a stationary point,
which would hopefully be a local minimum.

\section{Rate of convergence}

When descent direction is $-\grad_f(x^{(i)})$,
for both the Wolfe condition and the Goldstein condition,
the sum $\sum_{i=0}^{T-1} \left\| \grad_f(x^{(i)}) \right\|^2$ is upper-bounded.
Denote the upper bound by $N$.

Let $\delta = \min_i \|\grad_f(x^{(i)})\|$.
Then $T\delta^2 \le N$. Therefore, $\delta \le \sqrt{\frac{N}{T}}$.
This tells us how fast $x^{(i)}$ converges to a stationary point.

\end{document}
