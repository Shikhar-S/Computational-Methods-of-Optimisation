\documentclass[a4paper, 12pt, fleqn]{article}

\usepackage{amsmath}
\usepackage{amsthm}
\usepackage{amssymb}
\usepackage{xcolor}
\usepackage{float}
\usepackage{hyperref}
\usepackage{url}
\usepackage[margin=1in]{geometry}

\hypersetup{
    colorlinks,
    linkcolor={red!50!black},
    citecolor={red!50!black},
    urlcolor={blue!50!black}
}

\newtheorem{theorem}{Theorem}[section]
\newtheorem{definition}{Definition}[section]
\newtheorem{example}{Example}[section]
\newtheorem{corollary}{Corollary}[theorem]
\newtheorem{lemma}[theorem]{Lemma}

\newcommand*{\floor}[1]{\left\lfloor #1 \right\rfloor}
\newcommand*{\ceil}[1]{\left\lceil #1 \right\rceil}

\newcommand{\initFromContents}{
\tableofcontents
\newpage
\setlength{\parindent}{0pt}
\setlength{\parskip}{0.5em}
}

\newcommand{\initAfterBeginDocument}{
\maketitle
\initFromContents{}
}

\newcommand{\addMyBib}{
\bibliographystyle{plainurl}
\bibliography{bibdb}
}

\date{}


\title{CMO Lecture 3 notes}

\begin{document}

\maketitle
\initMinimal{}

$C^k$ is the set of all functions which are $k$-times differentiable
and whose $k^{\textrm{th}}$ derivative is continuous.

\[ \nabla(f)(x) = \left[ \frac{\partial f(x)}{\partial x_i} \right]_{i=1}^d \]

Multivariate $C^1$ means $\nabla(f)$ exists and all components are continuous.

$g(t) = f(x + tu)$. Here $g: \mathbb{R} \mapsto \mathbb{R}$.

\textbf{Homework}: Prove that $\frac{dg(t)}{dt} = (\nabla f(x+tu))^Tu$.

\begin{theorem} $f \in C^1 \implies g \in C^1$ \end{theorem}

$u = y - x$. $g(0) = f(x)$. $g(1) = f(y)$.

$g(t) = g(0) + g'(0)t + o(t)$.

$f(y) = f(x) + \nabla(f(x))^T(y-x) + o(\|y-x\|)$.

\[ \operatorname{Hess}(f)(x) = \left[ \frac{\partial^2 f(x)}{\partial x_i \partial x_j} \right]_{i, j \in [n]} \]

\textbf{Homework}: Prove that $g^{(2)}(t) = u^T H(f)(x+tu) u$.

When hessian is continuous, it is also symmetric.

$x^*$ is a local minimum of $f$ iff
$\exists r > 0, \forall x' \in S \cap N_r(x^*), f(x^*) \le f(x')$.

\end{document}
