\documentclass[a4paper, 12pt, fleqn]{article}

\usepackage{amsmath}
\usepackage{amsthm}
\usepackage{amssymb}
\usepackage{xcolor}
\usepackage{float}
\usepackage{hyperref}
\usepackage{url}
\usepackage[margin=1in]{geometry}

\hypersetup{
    colorlinks,
    linkcolor={red!50!black},
    citecolor={red!50!black},
    urlcolor={blue!50!black}
}

\newtheorem{theorem}{Theorem}[section]
\newtheorem{definition}{Definition}[section]
\newtheorem{example}{Example}[section]
\newtheorem{corollary}{Corollary}[theorem]
\newtheorem{lemma}[theorem]{Lemma}

\newcommand*{\floor}[1]{\left\lfloor #1 \right\rfloor}
\newcommand*{\ceil}[1]{\left\lceil #1 \right\rceil}

\newcommand{\initFromContents}{
\tableofcontents
\newpage
\setlength{\parindent}{0pt}
\setlength{\parskip}{0.5em}
}

\newcommand{\initAfterBeginDocument}{
\maketitle
\initFromContents{}
}

\newcommand{\addMyBib}{
\bibliographystyle{plainurl}
\bibliography{bibdb}
}

\date{}

\input{src/cmo/common.texlib}

\title{CMO: Conjugate Descent}

\begin{document}

\maketitle
\initMinimal{}

\textbf{Objective}: Minimize $f(x) = \frac{1}{2}x^TQx - b^Tx$,
where $Q$ is symmetric and positive definite.

\tableofcontents

\section{\texorpdfstring{$Q$}{Q}-conjugate vectors}

\begin{definition}
A set of $d$-dimensional non-0 vectors $U = \{u_0, u_1, \ldots, u_{k-1}\}$ is $Q$-conjugate
iff $\forall i \neq j, u_i^TQu_j = 0$.
\end{definition}

\begin{theorem}
If $U = \{u_0, \ldots, u_{d-1}\}$ is $Q$-conjugate, then $U$ is a basis of $\mathbb{R}^d$.
\end{theorem}
\begin{proof}
Assume $U$ is linearly dependent.
Then one of the vectors in $U$ can be represented as a linear combination of the other
(\href{https://sharmaeklavya2.github.io/theoremdep/nodes/linear-algebra/vector-spaces/linindep.html}{proof}).
Without loss of generality, assume $u_{d-1} = \sum_{i=0}^{d-2} \alpha_i u_i$.

$\forall i \neq d-1$,
\[ 0 = u_i^TQu_{d-1}
= u_i^TQ\left(\sum_{j=0}^{d-2} \alpha_j u_j \right)
= \sum_{j=0}^{d-2} \alpha_j u_i^TQu_j
= \alpha_i u_i^TQu_i
\implies \alpha_i = 0 \]
Hence, $u_{d-1} = 0 \Rightarrow \bot$.

On assuming $U$ to be linearly dependent, we got a contradiction.
Therefore, $U$ is linearly independent.

Since $|U| = d = \dim(\mathbb{R}^d)$,
$U$ is a basis of $\mathbb{R}^d$
(\href{https://sharmaeklavya2.github.io/theoremdep/nodes/linear-algebra/vector-spaces/basis/n-linindep-is-basis.html}{proof}).
\end{proof}

Since $Q$ is positive definite, $u_i^TQu_i > 0$ for all $i$.

\section{Descent algorithm using \texorpdfstring{$Q$}{Q}-conjugate vectors}

We'll develop a descent algorithm which uses $u_k$ in the $k^{\textrm{th}}$ iteration
with exact line search. The name of this algorithm will be `Conjugate Gradient Algorithm'.

Let $g(\alpha) = f(x_k + \alpha u_k)$ and $g_k = \grad_f(x_k)^T$
(sorry for overloading variables; the subscript will help distinguish them though).
Therefore, $g'(0) = \grad_f(x_k) = g_k$ and $g''(0) = u_k^TQu_k$.

By univariate Taylor series, we get
\[ g(\alpha) = g(0) + \alpha g'(0) + \frac{\alpha^2}{2} g''(0) \]

Let $\alpha_k^* = \operatorname{argmin}_{\alpha} f(x_k + \alpha u_k)$.
Therefore, \[ \alpha_k^* = - \frac{g'(0)}{g''(0)} = - \frac{g_k^Tu_k}{u_k^TQu_k} \]
We'll choose $x_{k+1} = x_k + \alpha_k^*u_k$.
Therefore, $x_k = x_0 + \sum_{i=0}^{k-1} \alpha_i^* u_i$.

\section{Proof of convergence}

\begin{theorem}
\label{thm:ug}
\[ u_j^T g_k = \begin{cases}0 & \textrm{ if } j < k \\ u_j^Tg_0 & \textrm{ if } j \ge k \end{cases} \]
\end{theorem}
\begin{proof}
\begin{align*}
g_k &= \grad_f(x_k) = Qx_k - b
\\ &= Q\left(x_0 + \sum_{i=0}^{k-1} \alpha_i^* u_i\right) - b
\\ &= (Qx_0 - b) + \sum_{i=0}^{k-1} \alpha_i^* Qu_i
\\ &= g_0 + \sum_{i=0}^{k-1} \alpha_i^* Qu_i
\end{align*}
\begin{align*}
u_j^Tg_k &= u_j^T\left( g_0 + \sum_{i=0}^{k-1} \alpha_i^* Qu_i \right)
\\ &= u_j^Tg_0 + \sum_{i=0}^{k-1} \alpha_i^* u_j^TQu_i
\\ &= u_j^Tg_0 + \sum_{i=0}^{k-1} \alpha_i^*
    \begin{Bmatrix} u_j^TQu_j & i = j \\ 0 & i\neq j \end{Bmatrix}
\\ &= u_j^Tg_0 + \begin{Bmatrix} \alpha_j^* u_j^TQu_j & j < k \\ 0 & j \ge k \end{Bmatrix}
\\ &= u_j^Tg_0 - \begin{Bmatrix} u_j^Tg_j & j < k \\ 0 & j \ge k \end{Bmatrix}
\end{align*}
When $j = k$, we get $u_k^Tg_k = u_k^Tg_0$. Therefore,
\begin{align*}
u_j^Tg_k &= u_j^Tg_0 - \begin{Bmatrix} u_j^Tg_j & j < k \\ 0 & j \ge k \end{Bmatrix}
\\ &= u_j^Tg_0 - \begin{Bmatrix} u_j^Tg_0 & j < k \\ 0 & j \ge k \end{Bmatrix}
\\ &= \begin{Bmatrix} 0 & j < k \\ u_j^Tg_0 & j \ge k \end{Bmatrix}
\end{align*}
\end{proof}

Let $B_k = \{x_0 + \sum_{i=0}^{k-1} \beta_i u_i : \beta_i \in \mathbb{R} \}$.
Since $U$ is a basis of $\mathbb{R}^d$, $B_d = \mathbb{R}^d$.
Therefore, to prove convergence of this algorithm,
we'll prove that $\forall k, x_k = \operatorname{argmin}_{x \in B_k} f(x)$.

$x_k = x_0 + \sum_{i=0}^{k-1} \alpha_i^* u_i$.
Let $\alpha^* = [\alpha_0^*, \ldots, \alpha_{k-1}^*]$.
Let $h(\beta) = f(x_0 + \sum_{i=0}^{k-1} \beta_i u_i)$.
Then $\min_{x \in B_k} f(x) = \min_{\beta \in \mathbb{R}^k} h(\beta)$.
Since $h(\alpha^*) = f(x_k)$, if we prove that
$\alpha^* = \operatorname{argmin}_{\beta \in \mathbb{R}^k} h(\beta)$,
then $x_k = \operatorname{argmin}_{x \in B_k} f(x)$.

\begin{theorem} $h(\beta)$ is a convex function. \end{theorem}
\begin{proof}
Let $U = [u_0, u_1, \ldots, u_{k-1}]$ be a $d$ by $k$ matrix. Then
\[ (U\beta)_j = \sum_{i=0}^{k-1} U[j, i] \beta_i
= \sum_{i=0}^{k-1} (u_i)_j \beta_i = \left( \sum_{i=0}^{k-1} u_i \beta_i \right)_j \]
\[ \implies h(\beta) = f\left(x_0 + \sum_{i=0}^{k-1} \beta_i u_i \right) = f(x_0 + U\beta) \]

\begin{align*}
h(\beta) &= f(x_0 + U\beta)
\\ &= f(x_0) + \grad_f(x_0)^T(U\beta) + \frac{1}{2}(U\beta)^T Q (U\beta) \tag{by Taylor series}
\\ &= f(x_0) + (\grad_f(x_0)^TU)\beta + \frac{1}{2}\beta^T (U^TQU) \beta
\end{align*}
This is a quadratic function in $\beta$.
It is convex iff $U^TQU$ is positive definite.

By the \href{https://sharmaeklavya2.github.io/theoremdep/nodes/linear-algebra/matrices/stacking/product.html}
{rules for multiplying stacked matrices}, we get that $(U^TQU)_{i,j} = u_i^TQu_j$.
Since vectors in $U$ are $Q$-conjugate, $u_i^TQu_j = 0$ when $i \neq j$.
Therefore, $U^TQU$ is a diagonal matrix.
Also, $\forall i, u_i^TQu_i > 0$ because $Q$ is positive definite.
Therefore, all diagonal entries of $U^TQU$ are positive.
Therefore, $U^TQU$ is positive definite.
\end{proof}

Since $h(\beta)$ is convex, $\grad_h(\beta) = 0$ is a necessary and sufficient condition for minimum.

For all $j \in [0, k]$
\[ h(\beta)_j = \frac{\partial f(x_0 + \sum_{i=0}^{k-1} \beta_i u_i)}{\partial \beta_j}
= u_j^T \grad_f\left(x_0 + \sum_{i=0}^{k-1} \beta_i u_i \right) \]
\[ h(\alpha^*)_j = u_j^T \grad_f\left(x_0 + \sum_{i=0}^{k-1} \alpha_i^* u_i \right)
= u_j^T \grad_f(x_k) = u_j^Tg_k = 0 \tag{by theorem \ref{thm:ug}} \]

Therefore, $\alpha^*$ minimizes $h$, so $x_d$ minimizes $f$.

\section{Rate of convergence}

Unlike the previous algorithms, this algorithm:
\begin{itemize}
\item Converges exactly (instead of only `approaching' the solution).
\item Converges very fast -- in exactly $d$ steps.
\end{itemize}

\section{Choosing \texorpdfstring{$Q$}{Q}-conjugate pairs}

We will find $U$ as follows:
$u_0 = -g_0$ and $u_{k+1} = -g_{k+1} + \beta_k u_k$.
We'll choose $\beta_k$ such that $u_k^TQu_{k+1} = 0$.
\[ 0 = u_k^TQu_{k+1} = -u_k^TQg_{k+1} + \beta_k u_k^TQu_k
\implies \beta_k = \frac{u_k^TQg_{k+1}}{u_k^TQu_k} \]

\begin{theorem}
$U$ is $Q$-conjugate.
\end{theorem}
\begin{proof}
Proof can be found in the
\href{https://www2.isye.gatech.edu/~nemirovs/Lect_OptII.pdf}{
lecture notes for the course
`Optimization II - Numerical Methods for Nonlinear Continuous Optimization'}
by A. Nemirovski, in Theorem 5.4.1, page 95.
\end{proof}

\begin{proof}[Proof sketch]
First induct on $k$ to prove that for all $k$,
\[ \operatorname{span}(\{g_0, g_1, \ldots, g_k\})
= \operatorname{span}(\{g_0, Qg_0, \ldots, Q^kg_0\})
= \operatorname{span}(\{u_0, u_1, \ldots, u_k\}) \]
This can be done using the facts that $g_{k+1} - g_k = Q(x_{k+1} - x_k) = \alpha_kQu_k$
and that $v_{k+1} = -g_{k+1} + \beta_k v_k$.

Then induct on $k$ to prove that
\[ \forall k, \forall i < k, u_k^TQu_i = 0 \]
To do this, express $v_{k+1}$ as $-g_{k+1} + \beta_k v_k$,
write $Qv_i$ as a linear combination of $\{v_0, v_1, \ldots, v_{i+1}\}$
and invoke theorem \ref{thm:ug}.
\end{proof}

\end{document}
