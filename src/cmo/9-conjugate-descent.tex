\documentclass[a4paper, 12pt, fleqn]{article}

\usepackage{amsmath}
\usepackage{amsthm}
\usepackage{amssymb}
\usepackage{xcolor}
\usepackage{float}
\usepackage{hyperref}
\usepackage{url}
\usepackage[margin=1in]{geometry}

\hypersetup{
    colorlinks,
    linkcolor={red!50!black},
    citecolor={red!50!black},
    urlcolor={blue!50!black}
}

\newtheorem{theorem}{Theorem}[section]
\newtheorem{definition}{Definition}[section]
\newtheorem{example}{Example}[section]
\newtheorem{corollary}{Corollary}[theorem]
\newtheorem{lemma}[theorem]{Lemma}

\newcommand*{\floor}[1]{\left\lfloor #1 \right\rfloor}
\newcommand*{\ceil}[1]{\left\lceil #1 \right\rceil}

\newcommand{\initFromContents}{
\tableofcontents
\newpage
\setlength{\parindent}{0pt}
\setlength{\parskip}{0.5em}
}

\newcommand{\initAfterBeginDocument}{
\maketitle
\initFromContents{}
}

\newcommand{\addMyBib}{
\bibliographystyle{plainurl}
\bibliography{bibdb}
}

\date{}

\input{src/cmo/common.texlib}
\usepackage{algorithm}
\usepackage{algpseudocode}

\title{CMO: Conjugate Descent}

\begin{document}

\maketitle
\initMinimal{}

\textbf{Objective}: Minimize $f(x) = \frac{1}{2}x^TQx - b^Tx$,
where $Q$ is symmetric and positive definite.

\tableofcontents

\section{\texorpdfstring{$Q$}{Q}-conjugate vectors}

\begin{definition}
A set of $d$-dimensional non-0 vectors $U = \{u_0, u_1, \ldots, u_{k-1}\}$ is $Q$-conjugate
iff $\forall i \neq j, u_i^TQu_j = 0$.
\end{definition}

\begin{theorem}
If $U = \{u_0, \ldots, u_{d-1}\}$ is $Q$-conjugate, then $U$ is a basis of $\mathbb{R}^d$.
\end{theorem}
\begin{proof}
Assume $U$ is linearly dependent.
Then one of the vectors in $U$ can be represented as a linear combination of the other
(\href{https://sharmaeklavya2.github.io/theoremdep/nodes/linear-algebra/vector-spaces/linindep.html}{proof}).
Without loss of generality, assume $u_{d-1} = \sum_{i=0}^{d-2} \alpha_i u_i$.

$\forall i \neq d-1$,
\[ 0 = u_i^TQu_{d-1}
= u_i^TQ\left(\sum_{j=0}^{d-2} \alpha_j u_j \right)
= \sum_{j=0}^{d-2} \alpha_j u_i^TQu_j
= \alpha_i u_i^TQu_i
\implies \alpha_i = 0 \]
Hence, $u_{d-1} = 0 \Rightarrow \bot$.

On assuming $U$ to be linearly dependent, we got a contradiction.
Therefore, $U$ is linearly independent.

Since $|U| = d = \dim(\mathbb{R}^d)$,
$U$ is a basis of $\mathbb{R}^d$
(\href{https://sharmaeklavya2.github.io/theoremdep/nodes/linear-algebra/vector-spaces/basis/n-linindep-is-basis.html}{proof}).
\end{proof}

Since $Q$ is positive definite, $u_i^TQu_i > 0$ for all $i$.

\section{Descent algorithm using \texorpdfstring{$Q$}{Q}-conjugate vectors}

We'll develop a descent algorithm which uses $u_k$ in the $k^{\textrm{th}}$ iteration
with exact line search. The name of this algorithm will be `Conjugate Gradient Algorithm'.

Let $g(\alpha) = f(x_k + \alpha u_k)$ and $g_k = \grad_f(x_k)^T$
(sorry for overloading variables; the subscript will help distinguish them though).
Therefore, $g'(0) = \grad_f(x_k) = g_k$ and $g''(0) = u_k^TQu_k$.

By univariate Taylor series, we get
\[ g(\alpha) = g(0) + \alpha g'(0) + \frac{\alpha^2}{2} g''(0) \]

Let $\alpha_k^* = \operatorname{argmin}_{\alpha} f(x_k + \alpha u_k)$.
Therefore, \[ \alpha_k^* = - \frac{g'(0)}{g''(0)} = - \frac{g_k^Tu_k}{u_k^TQu_k} \]
We'll choose $x_{k+1} = x_k + \alpha_k^*u_k$.
Therefore, $x_k = x_0 + \sum_{i=0}^{k-1} \alpha_i^* u_i$.

\section{Proof of convergence}

\begin{theorem}
\label{thm:ug}
\[ u_j^T g_k = \begin{cases}0 & \textrm{ if } j < k \\ u_j^Tg_0 & \textrm{ if } j \ge k \end{cases} \]
\end{theorem}
\begin{proof}
\begin{align*}
g_k &= \grad_f(x_k) = Qx_k - b
\\ &= Q\left(x_0 + \sum_{i=0}^{k-1} \alpha_i^* u_i\right) - b
\\ &= (Qx_0 - b) + \sum_{i=0}^{k-1} \alpha_i^* Qu_i
\\ &= g_0 + \sum_{i=0}^{k-1} \alpha_i^* Qu_i
\end{align*}
\begin{align*}
u_j^Tg_k &= u_j^T\left( g_0 + \sum_{i=0}^{k-1} \alpha_i^* Qu_i \right)
\\ &= u_j^Tg_0 + \sum_{i=0}^{k-1} \alpha_i^* u_j^TQu_i
\\ &= u_j^Tg_0 + \sum_{i=0}^{k-1} \alpha_i^*
    \begin{Bmatrix} u_j^TQu_j & i = j \\ 0 & i\neq j \end{Bmatrix}
\\ &= u_j^Tg_0 + \begin{Bmatrix} \alpha_j^* u_j^TQu_j & j < k \\ 0 & j \ge k \end{Bmatrix}
\\ &= u_j^Tg_0 - \begin{Bmatrix} u_j^Tg_j & j < k \\ 0 & j \ge k \end{Bmatrix}
\end{align*}
When $j = k$, we get $u_k^Tg_k = u_k^Tg_0$. Therefore,
\begin{align*}
u_j^Tg_k &= u_j^Tg_0 - \begin{Bmatrix} u_j^Tg_j & j < k \\ 0 & j \ge k \end{Bmatrix}
\\ &= u_j^Tg_0 - \begin{Bmatrix} u_j^Tg_0 & j < k \\ 0 & j \ge k \end{Bmatrix}
\\ &= \begin{Bmatrix} 0 & j < k \\ u_j^Tg_0 & j \ge k \end{Bmatrix}
\end{align*}
\end{proof}

\begin{corollary}
$g_d = 0$. This means that the conjugate descent algorithm converges in $d$ iterations.
\end{corollary}
\begin{proof}
By the previous theorem (\ref{thm:ug}), $\forall 0 \le j \le d-1, u_j^Tg_d = 0$.
Since $U = \{u_0, u_1, \ldots, u_{d-1}\}$ forms a basis of $\mathbb{R}^d$,
we get that $\forall x \in \mathbb{R}^d, x^Tg_d = 0$.
Therefore, $g_d^Tg_d = 0 \implies g_d = 0$.
\end{proof}

We'll now look at an alternative way of proving convergence
which will give us more insight.

Let $B_k = \{x_0 + \sum_{i=0}^{k-1} \beta_i u_i : \beta_i \in \mathbb{R} \}$.
Since $U$ is a basis of $\mathbb{R}^d$, $B_d = \mathbb{R}^d$.
Therefore, to prove convergence of this algorithm,
we'll prove the following theorem.

\begin{theorem}[Expanding subspace theorem]
\label{thm:exp-subsp}
$\forall k, x_k = \operatorname{argmin}_{x \in B_k} f(x)$.
\end{theorem}

$x_k = x_0 + \sum_{i=0}^{k-1} \alpha_i^* u_i$.
Let $\alpha^* = [\alpha_0^*, \ldots, \alpha_{k-1}^*]$.
Let $h(\beta) = f(x_0 + \sum_{i=0}^{k-1} \beta_i u_i)$.
Then $\min_{x \in B_k} f(x) = \min_{\beta \in \mathbb{R}^k} h(\beta)$.
Since $h(\alpha^*) = f(x_k)$, if we prove that
$\alpha^* = \operatorname{argmin}_{\beta \in \mathbb{R}^k} h(\beta)$,
then $x_k = \operatorname{argmin}_{x \in B_k} f(x)$.

\begin{lemma} $h(\beta)$ is a convex function. \end{lemma}
\begin{proof}
Let $U = [u_0, u_1, \ldots, u_{k-1}]$ be a $d$ by $k$ matrix. Then
\[ (U\beta)_j = \sum_{i=0}^{k-1} U[j, i] \beta_i
= \sum_{i=0}^{k-1} (u_i)_j \beta_i = \left( \sum_{i=0}^{k-1} u_i \beta_i \right)_j \]
\[ \implies h(\beta) = f\left(x_0 + \sum_{i=0}^{k-1} \beta_i u_i \right) = f(x_0 + U\beta) \]

\begin{align*}
h(\beta) &= f(x_0 + U\beta)
\\ &= f(x_0) + \grad_f(x_0)^T(U\beta) + \frac{1}{2}(U\beta)^T Q (U\beta) \tag{by Taylor series}
\\ &= f(x_0) + (\grad_f(x_0)^TU)\beta + \frac{1}{2}\beta^T (U^TQU) \beta
\end{align*}
This is a quadratic function in $\beta$.
It is convex iff $U^TQU$ is positive definite.

By the \href{https://sharmaeklavya2.github.io/theoremdep/nodes/linear-algebra/matrices/stacking/product.html}
{rules for multiplying stacked matrices}, we get that $(U^TQU)_{i,j} = u_i^TQu_j$.
Since vectors in $U$ are $Q$-conjugate, $u_i^TQu_j = 0$ when $i \neq j$.
Therefore, $U^TQU$ is a diagonal matrix.
Also, $\forall i, u_i^TQu_i > 0$ because $Q$ is positive definite.
Therefore, all diagonal entries of $U^TQU$ are positive.
Therefore, $U^TQU$ is positive definite.
\end{proof}

Since $h(\beta)$ is convex, $\grad_h(\beta) = 0$ is a necessary and sufficient condition for minimum.

For all $j \in [0, k-1]$
\[ h(\beta)_j = \frac{\partial f(x_0 + \sum_{i=0}^{k-1} \beta_i u_i)}{\partial \beta_j}
= u_j^T \grad_f\left(x_0 + \sum_{i=0}^{k-1} \beta_i u_i \right) \]
\[ h(\alpha^*)_j = u_j^T \grad_f\left(x_0 + \sum_{i=0}^{k-1} \alpha_i^* u_i \right)
= u_j^T \grad_f(x_k) = u_j^Tg_k = 0 \tag{by theorem \ref{thm:ug}} \]

Therefore, $\alpha^*$ minimizes $h$, so $x_d$ minimizes $f$.

\section{Rate of convergence}

Unlike the previous algorithms, this algorithm:
\begin{itemize}
\item Converges exactly (instead of only `approaching' the solution).
\item Converges very fast -- in exactly $d$ steps.
\end{itemize}

\section{Choosing \texorpdfstring{$Q$}{Q}-conjugate pairs}

We will find $U$ as follows:
$u_0 = -g_0$ and $u_{k+1} = -g_{k+1} + \beta_k u_k$.
We'll choose $\beta_k$ such that $u_k^TQu_{k+1} = 0$.
\[ 0 = u_k^TQu_{k+1} = -u_k^TQg_{k+1} + \beta_k u_k^TQu_k
\implies \beta_k = \frac{u_k^TQg_{k+1}}{u_k^TQu_k} \]

\begin{algorithm}[H]
\caption{$\operatorname{CGA}(x_0)$: Conjugate Gradient Algorithm for
$f(x) = \frac{1}{2}x^TQx - b^Tx$. Takes starting point as input.}
\label{alg:conj-grad-for-quadratic}
\begin{algorithmic}[1]
\State $g_0 = Qx_0 - b$
\If{$g_0 \texttt{ == } 0$}
    \State \Return $x_0$
\EndIf
\State $u_0 = -g_0$
\For{$i \in [0, \infty)$}
    \State $\displaystyle \alpha_i = \frac{-g_i^Tu_i}{u_i^TQu_i}$
    \State $x_{i+1} = x_i + \alpha_iu_i$
    \State $g_{i+1} = Qx_{i+1} - b$
    \If{$g_{i+1} \texttt{ == } 0$}
        \State \Return $x_{i+1}$
    \EndIf
    \State $\displaystyle \beta_i = \frac{u_i^TQg_{i+1}}{u_i^TQu_i}$
    \State $u_{i+1} = -g_{i+1} + \beta_iu_i$
\EndFor
\end{algorithmic}
\end{algorithm}

\begin{theorem}
$U$ is $Q$-conjugate.
\end{theorem}
\begin{proof}
Proof can be found in the
\href{https://www2.isye.gatech.edu/~nemirovs/Lect_OptII.pdf}{
lecture notes for the course
`Optimization II - Numerical Methods for Nonlinear Continuous Optimization'}
by A. Nemirovski, in Theorem 5.4.1, page 95.
\end{proof}

\begin{proof}[Proof sketch]
First induct on $k$ to prove that for all $k$,
\[ \operatorname{span}(\{g_0, g_1, \ldots, g_k\})
= \operatorname{span}(\{g_0, Qg_0, \ldots, Q^kg_0\})
= \operatorname{span}(\{u_0, u_1, \ldots, u_k\}) \]
This can be done using the facts that $g_{k+1} - g_k = Q(x_{k+1} - x_k) = \alpha_kQu_k$
and that $v_{k+1} = -g_{k+1} + \beta_k v_k$.

Then induct on $k$ to prove that
\[ \forall k, \forall i < k, u_k^TQu_i = 0 \]
To do this, express $v_{k+1}$ as $-g_{k+1} + \beta_k v_k$,
write $Qv_i$ as a linear combination of $\{v_0, v_1, \ldots, v_{i+1}\}$
and carefully invoke theorem \ref{thm:ug}.
\end{proof}

\section{Faster convergence for structured eigenvalues}

\newcommand*{\MatPoly}[1]{\textrm{Poly}^{#1}}

When the eigenvalues of $Q$ have certain properties,
we can guarantee faster convergence.

$B_{k+1} = x_0 + \operatorname{span}(u_0, \ldots, u_k)$.
Therefore, any vector $x \in B_{k+1}$ can be expressed as $x_0 + \sum_{i=0}^k \gamma_i u_i$.
Since $\operatorname{span}(u_0, \ldots, u_k) = \operatorname{span}(g_0, \ldots, Q^kg_0)$,
$x = x_0 + \left(\sum_{i=0}^k \delta_i Q^i\right)g_0$.

Let $\MatPoly{k}$ be the set of univariate polynomials of degree at most $k$
where the coefficients are from $\mathbb{R}$ and the variable is
an $n$ by $n$ matrix over $\mathbb{R}$. Therefore,
\[ x \in B_{k+1} \implies \left( \exists P_k \in \MatPoly{k}, x = x_0 + P_k(Q)g_0 \right) \]
\begin{align*}
x - x^* &= (x_0 - x^*) + P_k(Q)g_0
= (x_0 - x^*) + P_k(Q)Q(x_0 - x^*)
\\ &= (I + QP_k(Q))(x_0 - x^*)
\end{align*}
Define $E(x) = f(x) - f(x^*)$.
By Taylor series,
\begin{align*}
E(x) &= \frac{1}{2} (x-x^*)^TQ(x-x^*)
\\ &= \frac{1}{2} (x_0 - x^*)^T(I + QP_k(Q))^TQ(I+QP_k(Q))(x_0 - x^*)
\\ &= \frac{1}{2} (x_0 - x^*)^T Q(I + QP_k(Q))^2 (x_0 - x^*)
\end{align*}

Let $R = \{e_1, e_2, \ldots, e_d\}$ be the set of orthonormal eigenvectors of $Q$.
Let $\lambda_1 \ge \lambda_2 \ge \ldots \ge \lambda_d$ be the corresponding eigenvalues.
Since $R$ forms a basis of $\mathbb{R}^d$, $x_0 - x^*$ can be represented as a linear combination of $R$.
Let $x_0 - x^* = \sum_{i=1}^d \zeta_i e_i = \zeta_i$.

\begin{lemma}
$E(x_0) = \frac{1}{2} \sum_{i=1}^d \zeta_i^2 \lambda_i$
\end{lemma}
\begin{proof}
Let $R$ be a matrix whose $i^{\textrm{th}}$ column is $e_i$.
Since the eigenvectors are orthonormal, $RR^T = R^TR = I$.
Let $\zeta = [\zeta_1, \ldots, \zeta_d]^T$. Then
\[ R\zeta = \sum_{i=1}^d \zeta_i e_i = x_0 - x^* \]

Since $Q$ is symmetric, $Q = RDR^T$,
Where $D$ is a diagonal matrix whose $i^{\textrm{th}}$ entry is $\lambda_i$.
Therefore,
\begin{align*}
2E(x_0) &= (x_0 - x^*)^T Q (x_0 - x^*) = (R\zeta)^T(RDR^T)(R\zeta)
\\ &= \zeta^T (R^TR) D (R^TR) \zeta = \zeta^T D \zeta = \sum_{i=1}^d \zeta_i^2 \lambda_i
\end{align*}
\end{proof}

\begin{lemma}[Homework]
Let $T$ be a polynomial where $T(X) = X(I + XP_k(X))^2$. Then
$E(x) = \frac{1}{2} \sum_{i=1}^d \zeta_i^2 T(\lambda_i)$.
\end{lemma}
\begin{proof}[Hint]
Use the fact that for all $j \in \mathbb{N}$,
$R$ is also the set of eigenvectors of $Q^j$
and the corresponding eigenvalues are $\lambda_1^j, \ldots, \lambda_d^j$.
\end{proof}

\begin{lemma}
For any polynomial $P_k \in \MatPoly{k}$,
\[ \frac{E(x_{k+1})}{E(x_0)} \le \max_{i=0}^d (1 + \lambda_i P_k(\lambda_i))^2 \]
\end{lemma}
\begin{proof}
\begin{align*}
E(x_{k+1}) &= \min_{x \in B_{k+1}} E(x)  \tag{Expanding subspace theorem}
\\ &= \min_{P_k \in \MatPoly{k}} \frac{1}{2} \sum_{i=1}^d \zeta_i^2 \lambda_i(1 + \lambda_iP_k(\lambda_i))^2
\\ &\le \min_{P_k \in \MatPoly{k}} \frac{1}{2} \sum_{i=1}^d \left(\zeta_i^2 \lambda_i
    \left(\max_{i=0}^d (1 + \lambda_i P_k(\lambda_i))^2\right)\right)
\\ &= \min_{P_k \in \MatPoly{k}} \left( \frac{1}{2} \sum_{i=1}^d \zeta_i^2 \lambda_i \right)
    \left(\max_{i=0}^d (1 + \lambda_i P_k(\lambda_i))^2\right)
\\ &= E(x_0) \min_{P_k \in \MatPoly{k}} \max_{i=0}^d (1 + \lambda_i P_k(\lambda_i))^2
\end{align*}
\end{proof}

Therefore, by cleverly choosing a polynomial, we can prove useful bounds on convergence.

\subsection{\texorpdfstring{$Q$}{Q} has \texorpdfstring{$r$}{r} distinct eigenvalues}

Suppose $Q$ has $r$ distinct eigenvalues $\mu_1 > \mu_2 > \ldots > \mu_r$.
Let $\overline{P}_r(x) = 1 + x P_{r-1}(x)$.

We'll construct $P_{r-1}$ such that $\overline{P}_r(x) = 0$ for all $1 \le i \le r$.
This would mean that $\frac{E(x_r)}{E(x_0)} = 0$, so
the conjugate gradient algorithm will converge in $r$ iterations.

Define $\overline{P}_r$ and $P_{r-1}$ as follows:
\begin{align*}
\overline{P}_r(x) &= \prod_{j=1}^r \left( 1 - \frac{x}{\mu_j} \right)
& P_{r-1}(x) &= \frac{\overline{P}_r(x)-1}{x}
\end{align*}

\begin{lemma}
\label{thm:p-is-poly}
$P_{r-1}$ is a polynomial of degree $r-1$ such that $\forall 0 \le i \le r, \overline{P}_r(\mu_i) = 0$.
\end{lemma}
\begin{proof}
Clearly, $\overline{P}_r(\mu_i) = 0$ for all $i$. Also, the degree of $\overline{P}$ is $r$.

Next, we must prove that $P_{r-1}$ is a polynomial.
Note that $\overline{P}_r(0) = 1$, so $0$ is a root of $\overline{P}_r(x) - 1$.
Therefore, $x$ is a factor of $\overline{P}_r(x)-1$
and hence $P_{r-1}$ is a polynomial.

Since the degree of $\overline{P}_r$ is $r$, the degree of $P_{r-1}$ is $r-1$.
\end{proof}

\subsection{Theorem for a polynomial}

In this section, we'll prove a theorem for a certain polynomial which we'll use in the next section.

\begin{theorem}
\label{thm:prod-poly}
Let $n \ge 2$. Let $0 < a_1 < a_2 < \ldots < a_n$.
Let $p_1, p_2, \ldots, p_n$ be positive integers and let $p_1 = 1$.
\begin{align*}
f(x) &= \prod_{i=1}^n \left( 1 - \frac{x}{a_i} \right)^{p_i} &
g(x) &= f(x) - 1 + \frac{x}{a_1}
\end{align*}
%Let $\eta$ be the leftmost root of $f'(x)$.
Then
\begin{enumerate}
\item \label{thm-part:f-val} $f$ is positive in $(-\infty, a_1)$, negative in $(a_1, a_2)$ and 0 at $a_1$ and $a_2$.
%\item $\eta \in (a_1, a_2)$.
%\item \label{thm-part:d-val} $f'$ is negative in $(-\infty, \eta)$ and positive in $(\eta, a_2)$.
%\item \label{thm-part:d2-val} $f''$ is convex in $(-\infty, \eta]$.
\item \label{thm-part:g-val} $g(x) \le 0$ for $x \in [0, a_1]$ and $g(x) \ge 0$ for $x \in [a_1, a_2]$.
\end{enumerate}
\end{theorem}
\begin{proof}
Since $a_1$ and $a_2$ are zeros of $f$, $f(a_1) = f(a_2) = 0$.
Since $a_1$ is the leftmost zero of $f$,
$f$ has the same sign in $(-\infty, a_1)$ (by intermediate value theorem).
Since $f(0) = 1$, $f$ is positive in $(-\infty, a_1)$.

\[ \frac{f'(x)}{f(x)} = \sum_{i=1}^n \frac{p_i}{x - a_i} \]
Let
\[ h_1(x) = \prod_{i=1}^n (x - a_i)^{p_i - 1} \]
Then $h_1(x)$ divides $f'(x)$.

By Rolle's theorem, there must be points $b_1 < b_2 < \ldots < b_{n-1}$
such that for all $i$, $f'(b_i) = 0$ and $b_i \in (a_i, a_{i+1})$. Let
\[ h_2(x) = \prod_{i=1}^{n-1} (x - b_i) \]
So $h_2(x)$ divides $f'(x)$.

Let $N = \sum_{i=1}^n p_i$. Then $\deg(f) = N$. Also
\[ \deg(h_1 h_2) = \deg(h_1) + \deg(h_2) = (N-n) + (n-1) = N-1 = \deg(f') \]
Therefore, $f'(x) = \gamma h_1(x) h_2(x)$ for some $\gamma \in \mathbb{R}$.

Since $p_1 = 1$, $b_1$ is the leftmost zero of $f'$
and it is the only zero in $(-\infty, a_2)$.
Therefore, $f'(x)$ has the same sign for $x \in (-\infty, b_1)$.
Since $f(0) = 1$, $f'(0) = - \sum_{i=1}^n \frac{1}{a_i} < 0$.
Therefore, $f'(x) < 0$ for $x \in (-\infty, b_1)$.

Since $f(a_1) = 0$ and $f'(a_1) < 0$, $f(a_1 + \epsilon) < 0$ for all very small $\epsilon$.
Also, $f$ has the same sign in $(a_1, a_2)$, otherwise
it would have a root in $(a_1, a_2)$, which we know is false.
Therefore, $f(x) < 0$ for $x \in (a_1, a_2)$.
This completes the proof of part \ref{thm-part:f-val} of this theorem.

Applying Rolle's theorem to $f'(x)$ and by a similar argument (todo: expand this),
we get that $f''(x)$ must have its leftmost root in $(b_1, a_2)$.
Therefore, $f''(x)$ has the same sign in $(-\infty, b_1]$.
\[ \frac{f''(x)}{f(x)} = \left(\sum_{i=1}^n \frac{p_i}{a_i - x}\right)^2
    - \sum_{i=1}^n \frac{p_i}{(a_i - x)^2} \]
\[ \implies f''(0) = \left(\sum_{i=1}^n \frac{p_i}{a_i} \right)^2 - \sum_{i=1}^n \frac{p_i}{a_i^2} > 0 \]
Therefore, $f''(x) > 0$ for $x \in (-\infty, b_1]$.

$f'(b_1) = 0$ and $f''(b_1) > 0$.
Therefore, $f'(b_1 + \epsilon) > 0$ for all very small $\epsilon$.
$f'(x)$ has the same sign in $(b_1, a_2)$ because $b_1$ is the only root of $f'(x)$ in $[b_1, a_2)$.
Therefore, $f'(x) > 0$ for $x \in (b_1, a_2)$.

Since $f$ is convex in $(-\infty, b_1]$, for $\alpha \in [0, 1]$,
\[ f(\alpha a_1) = f((1-\alpha)0 + \alpha a_1)
\le (1-\alpha)f(0) + \alpha f(a_1) = (1-\alpha) \]
Setting $\alpha$ to $x / a_1$, we get that for $x \in [0, a_1]$,
$f(x) \le 1 - \frac{x}{a_1} \Rightarrow g(x) \le 0$.

$g(0) = g(a_1) = 0$. By Rolle's theorem,
$\exists x_0 \in (0, a_1), g'(x_0) = 0$.
Since $g''(x) = f''(x) > 0$ for $x \in (-\infty, b_1]$,
$g'(x) > 0$ for $x \in (x_0, b_1]$.

$g'(x) = f'(x) + \frac{1}{a_1}$.
For $x \in (b_1, a_2)$, $f'(x) > 0 \Rightarrow g'(x) > 0$.
Therefore, $g'(x) > 0$ for $x \in [a_1, b_1)$.

Since $g(a_1) = 0$ and $g'(x) > 0$ for $x \in [a_1, b_1)$,
$g(x) > 0$ for $x \in (a_1, b_1)$.
\end{proof}

\subsection{\texorpdfstring{$Q$}{Q} has some clustered eigenvalues}

Suppose $Q$ has eigenvalues $\lambda_1 \ge \lambda_2 \ge \ldots \ge \lambda_d$,
where for some constants $a$ and $b$,
\[ 0 < a \le \lambda_d \le \ldots \le \lambda_{r+1} < b < \lambda_r \le \ldots \le \lambda_1 \]

Let $\mu_i = \lambda_i$ for $i$ from 1 to $r$.
Let $\mu_{r+1} = \frac{a+b}{2}$.
\begin{align*}
\overline{P}_{r+1}(x) &= \prod_{j=1}^{r+1} \left( 1 - \frac{x}{\mu_j} \right)
& P_{r}(x) &= \frac{P_{r+1}(x) - 1}{x}
& h(x) &= 1 - \frac{x}{\mu_{r+1}}
\end{align*}

It's easy to prove (similar to lemma \ref{thm:p-is-poly}) that $P_{r}$ is a polynomial and has degree $r$.

Since $\overline{P}_{r+1}$ is of the right form, we can apply theorem \ref{thm:prod-poly}.

By part \ref{thm-part:f-val} of theorem \ref{thm:prod-poly},
we get that for $x \in [a, \frac{a+b}{2}]$, $\overline{P}_{r+1}(x) \ge 0$.
By part \ref{thm-part:g-val} of theorem \ref{thm:prod-poly},
we get that for $x \in [a, \frac{a+b}{2}]$,
\[ \overline{P}_{r+1}(x) \le h(x) \le h(a) = \frac{b-a}{b+a} \]

By part \ref{thm-part:f-val} of theorem \ref{thm:prod-poly},
we get that for $x \in [\frac{a+b}{2}, b]$, $\overline{P}_{r+1}(x) \le 0$.
By part \ref{thm-part:g-val} of theorem \ref{thm:prod-poly},
we get that for $x \in [\frac{a+b}{2}, b]$,
\[ \overline{P}_{r+1}(x) \ge h(x) \ge h(b) = - \frac{b-a}{b+a} \]

Therefore, for $x \in [a, b]$, $\left| \overline{P}_{r+1}(x) \right| \le \frac{b-a}{b+a}$.
Therefore,
\[ \frac{E(x_{r+1})}{E(x_0)} \le \left( \frac{b-a}{b+a} \right)^2 \]

We can use the above fact to design an algorithm called the `partial conjugate gradient' algorithm.
In this algorithm, we'll start at the point $z_0$ and run
the conjugate gradient algorithm for $r+1$ steps to reach the point $z_1$.
Then we'll rerun the conjugate gradient algorithm for $r+1$ steps from $z_1$ to reach a point $z_2$,
then we'll rerun the conjugate gradient algorithm for $r+1$ steps from $z_2$ to reach a point $z_3$, and so on.
We'll do this $l$ times. After $l$ iterations $\frac{E(z_l)}{E(z_0)} = \left( \frac{b-a}{b+a} \right)^{2l}$.
This will give us linear convergence.

\end{document}
