\documentclass[a4paper, 12pt, fleqn]{article}

\usepackage{amsmath}
\usepackage{amsthm}
\usepackage{amssymb}
\usepackage{xcolor}
\usepackage{float}
\usepackage{hyperref}
\usepackage{url}
\usepackage[margin=1in]{geometry}

\hypersetup{
    colorlinks,
    linkcolor={red!50!black},
    citecolor={red!50!black},
    urlcolor={blue!50!black}
}

\newtheorem{theorem}{Theorem}[section]
\newtheorem{definition}{Definition}[section]
\newtheorem{example}{Example}[section]
\newtheorem{corollary}{Corollary}[theorem]
\newtheorem{lemma}[theorem]{Lemma}

\newcommand*{\floor}[1]{\left\lfloor #1 \right\rfloor}
\newcommand*{\ceil}[1]{\left\lceil #1 \right\rceil}

\newcommand{\initFromContents}{
\tableofcontents
\newpage
\setlength{\parindent}{0pt}
\setlength{\parskip}{0.5em}
}

\newcommand{\initAfterBeginDocument}{
\maketitle
\initFromContents{}
}

\newcommand{\addMyBib}{
\bibliographystyle{plainurl}
\bibliography{bibdb}
}

\date{}

\input{src/cmo/common.texlib}

\usepackage{algorithm}
\usepackage{algpseudocode}

\title{CMO 1: Preliminaries}

\begin{document}

\maketitle
\initMinimal{}

\tableofcontents

\section{Central problem and algorithm template}

\textbf{Central Problem of the course `Computational Methods of Optimization'}:
Given an objective function $f: \mathbb{R}^d \mapsto \mathbb{R}$
and a constraint set $S \subseteq \mathbb{R}^d$,
find $x^* = \operatorname{argmin}_{x \in S} f(x)$ and $f^* = f(x^*)$.

Example: for $\min_{x \in \mathbb{R}} (x-t)^2$, $x^* = t$ and $f^* = 0$.

All algorithms we develop to find $x^*$ will follow this template:
\begin{algorithm}[H]
\label{algo:template}
\begin{algorithmic}
\State Pick $x \in S$.
\While{$x$ is not optimal}
    \State Pick another $x \in S$ such that $f(x)$ decreases.
\EndWhile
\State \Return $x$
\end{algorithmic}
\end{algorithm}

\section{Metric space}

For any set $S$ (we'll usually consider $S = \mathbb{R}^d$),
$D: S \times S \mapsto \mathbb{R}$ is a distance function
iff all of the following are true:
\begin{itemize}
\item $D(x, y) = 0 \iff x = y$.
\item $D(x, y) \ge 0$.
\item Symmetry: $D(x, y) = D(y, x)$.
\item Triangle inequality: $D(x, y) + D(y, z) \ge D(x, z)$.
\end{itemize}

\begin{theorem}$D(x, y) = \|x - y\|$ is a distance function. Here
\[ \|x\| = \sqrt{x^Tx} = \sqrt{\sum_{i=1}^d x_i^2} \]
\end{theorem}
\begin{theorem}$D(x, y) = \sum_{i=1}^d |x_i - y_i|$ is a distance function.\end{theorem}

\section{Neighborhood function and Open sets}

\begin{definition} For $r > 0$ and $x \in \mathbb{R}^d$, $N_r(x) = \{z: D(x, z) < r\}$
is called a neighborhood of $x$ of radius $r$. \end{definition}

\begin{definition} $x \in \mathbb{R}^d$ is an interior point of $S$ iff
$\exists r > 0, N_r(x) \subseteq S$. \end{definition}

\begin{definition}Let $x, y \in \mathbb{R}$.\begin{itemize}
\item $(x, y) = \{z: x < z < y\}$.
\item $(x, y] = \{z: x < z \le y\}$.
\item $[x, y) = \{z: x \le z < y\}$.
\item $[x, y] = \{z: x \le z \le y\}$.
\end{itemize}\end{definition}

\begin{definition} $S$ is an open set iff $\forall x \in S$,
$x$ is an interior point of $S$. \end{definition}

\begin{example} $(0, 1)$ is an open set but $[0, 1)$ is not. \end{example}

\begin{definition} $x \in \mathbb{R}^d$ is a limit point of $S$ iff
$N_r(x) \cap S \neq \phi$. \end{definition}

\begin{example} $0, \frac{1}{2}, 1$ are 3 of the limit points of $(0, 1]$. \end{example}

\begin{definition} Closure of a set $S$ is the set of all limit points of $S$. \end{definition}

\begin{definition} A set $S$ is closed iff all limit points of $S$ lie in $S$. \end{definition}

\begin{example} $[0, 1]$ is a closed set. \end{example}

\section{Limit and Bounds}

\begin{definition} Let $[x_i]_{i \in \mathbb{N}}$ be an infinite sequence
where $x \in \mathbb{R}^d$. Then
\[ \lim_{i \rightarrow \infty} x_i = x
\iff \forall \epsilon > 0, \exists n, \forall i \ge n, \|x - x_i\| < \epsilon \]
\end{definition}

\begin{definition} $S \subseteq \mathbb{R}^d$ is a bounded set
iff $\exists M, \forall x \in S, \|x\| \le M$. \end{definition}

\begin{definition} For $x_i \in \mathbb{R}$, $M$ is an upper bound of $[x_i]_{i \in \mathbb{N}}$
iff $\forall i, x_i \le M$. A sequence with an upper bound is called an upper-bounded sequence.
\end{definition}

\begin{definition} $g$ is a least upper bound (LUB) (of $[x_i]_{i \in \mathbb{N}}$) iff
$g$ is an upper bound and for every upper bound $h$, $g \le h$. \end{definition}

\begin{example} For $x_i = 1-\frac{1}{i}$, LUB is 1. \end{example}

\begin{theorem} A monotonic bounded sequence has a limit. \end{theorem}

\section{Continuity}

\begin{definition}
\[ \lim_{x\rightarrow p} f(x) = q \iff \forall \epsilon > 0, \exists \delta > 0,
\forall x \in N_{\delta}(p), f(x) \in N_{\epsilon}(q) \]
\end{definition}

\begin{definition}
$f$ is continuous at $x$ $\iff \lim_{x\rightarrow p} f(x) = f(p)$.
$f$ is continuous over $S \iff f$ is continuous at all points $x \in S$.
\end{definition}

\begin{theorem}
Let $S \subseteq \mathbb{R}^d$ be closed and bounded.
Let $f(S) = \{f(x): x \in S\}$.
Let $f$ be continuous over $S$.
Then $f(S)$ is closed and bounded.
\end{theorem}

For optimization problems, $x^*$ is guaranteed to exist iff
$f$ is continuous and $S$ is closed and bounded.
Henceforth, we will assume $S$ to be closed and bounded
and assume functions to be continuous.

\section{Asymptotics}

\[ a(x) \in o(b(x)) \iff \lim_{x \rightarrow x_0} \left| \frac{a(x)}{b(x)} \right| = 0\]

For example, at $x=0$, $x^3 \in o(x^2)$.

If $f$ is continuous at $x=p$, $f(x) = f(p) + o(1)$.

\end{document}
