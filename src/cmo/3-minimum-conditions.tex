\documentclass[a4paper, 12pt, fleqn]{article}

\usepackage{amsmath}
\usepackage{amsthm}
\usepackage{amssymb}
\usepackage{xcolor}
\usepackage{float}
\usepackage{hyperref}
\usepackage{url}
\usepackage[margin=1in]{geometry}

\hypersetup{
    colorlinks,
    linkcolor={red!50!black},
    citecolor={red!50!black},
    urlcolor={blue!50!black}
}

\newtheorem{theorem}{Theorem}[section]
\newtheorem{definition}{Definition}[section]
\newtheorem{example}{Example}[section]
\newtheorem{corollary}{Corollary}[theorem]
\newtheorem{lemma}[theorem]{Lemma}

\newcommand*{\floor}[1]{\left\lfloor #1 \right\rfloor}
\newcommand*{\ceil}[1]{\left\lceil #1 \right\rceil}

\newcommand{\initFromContents}{
\tableofcontents
\newpage
\setlength{\parindent}{0pt}
\setlength{\parskip}{0.5em}
}

\newcommand{\initAfterBeginDocument}{
\maketitle
\initFromContents{}
}

\newcommand{\addMyBib}{
\bibliographystyle{plainurl}
\bibliography{bibdb}
}

\date{}

\input{src/cmo/common.texlib}

\title{CMO: Existence and Characterization of Minimum}

\begin{document}

\maketitle
\initMinimal{}

Let $S \subseteq \mathbb{R}^d$ and $f: S \mapsto \mathbb{R}$.
\\ $x^*$ is a local minimum $\iff \exists r > 0, \forall x \in N_r(x^*) \cap S, f(x^*) \le f(x)$.

We'll restrict our analysis in 2 ways:
\begin{itemize}
\item We'll only consider functions for which a global minimum exists.
Here we'll discuss a sufficient condition for that.
\item We'll only try to find a local minimum, since finding global minimum is difficult.
\end{itemize}

\section{Necessary condition for local minimum of univariate function}

\begin{theorem}
If $f: \mathbb{R} \mapsto \mathbb{R}$ is differentiable,
then $x^*$ is the local minimum of $f \implies f'(x^*) = 0$.
\end{theorem}
\begin{proof}
Let
\[ h(t) = \frac{f(t) - f(x^*)}{t - x^*} \]
Then $f'(x^*) = \lim_{t \rightarrow x^*} h(t)$.

Suppose $x^*$ is a local minimum in $(x-r, x+r)$.
Then for $t \in (x-r, x)$, $h(t) \le 0$ and for $t \in (x, x+r)$, $h(t) \ge 0$.
Therefore, left derivative of $f$ at $x^*$ is non-positive
and right derivative of $f$ at $x^*$ is non-negative.
Since $f$ is differentiable, left and right derivatives are equal.
Therefore, $f'(x^*) = 0$.
\end{proof}

\begin{theorem} Let $f$ be a $C^2$ function and $x^*$ be a local minimum.
Then $f''(x^*) \ge 0$. \end{theorem}
\begin{proof}
Using Taylor series near $x^*$, we get
\[ f(x) = f(x^*) + (x-x^*)f'(x^*) + \frac{1}{2}(x-x^*)^2f''(x^*) + o((x-x^*)^2) \]
\[ \implies 0 \le f(x) - f(x^*) = \frac{1}{2}(x-x^*)^2f''(x^*) + o((x-x^*)^2) \]
For this to hold true for all $x$ near $x^*$, $f''(x^*) \ge 0$.
\end{proof}

\section{Characterization of functions which have a minimum}

Consider a function from $\mathbb{R}^d$ to $\mathbb{R}$.
Global minimum exists iff $f$ is lower-bounded.

\begin{definition}
\[ \lim_{\|x\| \rightarrow \infty} f(x) = \infty
\iff \forall F > 0, \exists M > 0, \forall x \in \mathbb{R}^d,
(\|x\| > M \implies f(x) \ge F) \]
If $\lim_{\|x\| \rightarrow \infty} f(x) = \infty$, then $f$ is called a \textbf{coercive} function.
\end{definition}

\begin{theorem}[Weierstrass' theorem]
If a continuous function's domain is closed and bounded,
the function has a global minimum and maximum.
\end{theorem}

\begin{theorem}
\[ \lim_{\|x\| \rightarrow \infty} f(x) = \infty \wedge f \textrm{ is continuous }
\implies f \textrm{ has global minimum} \]
\end{theorem}
\begin{proof}
Consider $F = f(0)$.
Let $S_1 = \{x: \|x\| > M \}$ and $S_2 = \{x: \|x\| \le M \}$.

Since $f$ is coercive, $\forall x \in S_1, f(0) \le f(x)$.
By Weierstrass' theorem, a global minimum exists in $S_2$. Let it be $x^*$.
Therefore, $f(x^*) \le f(0)$.
Therefore, $x^*$ is a global minimum of $\mathbb{R}^d$.
\end{proof}

\section{Sufficient condition for local minimum of univariate function}

\begin{theorem}
$f'(x_0) = 0 \wedge f''(x_0) > 0 \implies x_0 \textrm{ is local minimum}$.
\end{theorem}
\begin{proof}
\[ f(x) - f(x_0) = \frac{1}{2}(x-x^*)^2f''(x^*) + o((x-x^*)^2) \]
In the neighborhood of $x_0$, the small-o term is negligible,
so the $f''(x^*)$ makes $f(x) - f(x_0)$ positive.
Therefore, $x_0$ is a local minimum in that neighborhood.
\end{proof}

\section{Necessary condition for local minimum of multivariate function}

\begin{theorem} Let $f: \mathbb{R}^d \mapsto \mathbb{R}$ be a differentiable function.
Let $x^*$ be a local minimum of $f$. Then $\grad_f(x^*) = 0$.
\end{theorem}
\begin{proof}
Let $u \in \mathbb{R}^d$ and $t \in \mathbb{R}$.
\[ x^* + tu \in N_r(x^*) \iff \|tu\| < r \iff |t| \le \frac{r}{\|u\|} \iff t \in N_{\frac{r}{\|u\|}}(0) \]
Let $g(t) = f(x^* + tu)$.
\begin{align*}
& x^* \textrm{ is local minimum of } f
\\ &\Rightarrow \forall x \in N_r(x^*), f(x^*) \le f(x)
\\ &\Rightarrow \forall t \in N_{\frac{r}{\|u\|}}(0), f(x^*) \le f(x^* + tu)
\\ &\Rightarrow \forall t \in N_{\frac{r}{\|u\|}}(0), g(0) \le g(t)
\\ &\Rightarrow g \textrm{ has local minimum at } 0
\\ &\Rightarrow g'(0) = 0
\\ &\Rightarrow \grad_f(x^*)^Tu = 0
\end{align*}
Since this is true for all $u \in \mathbb{R}^d$, $\grad_f(x^*) = 0$.
\end{proof}

\begin{theorem} Let $f: \mathbb{R}^d \mapsto \mathbb{R}$ be a differentiable function.
Let $x^*$ be a local minimum of $f$. Then $\hessian_f(x^*)$ is positive semi-definite.
\end{theorem}
\begin{proof}
Similar to above proof. Use the fact that if $g$ has a local minimum at 0, then $g''(0) \ge 0$.
\end{proof}

\section{Sufficient condition for local minimum of multivariate function}

\begin{theorem} Let $f: \mathbb{R}^d \mapsto \mathbb{R}$ be a differentiable function.
Let $\grad_f(x_0) = 0$ and $\hessian_f(x_0)$ be positive definite.
Then $x_0$ is a local minimum of $f$.
\end{theorem}
\begin{proof} Proof follows directly from Taylor series. \end{proof}

\end{document}
