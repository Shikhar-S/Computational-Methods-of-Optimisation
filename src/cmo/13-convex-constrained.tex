\documentclass[a4paper, 12pt, fleqn]{article}

\usepackage{amsmath}
\usepackage{amsthm}
\usepackage{amssymb}
\usepackage{xcolor}
\usepackage{float}
\usepackage{hyperref}
\usepackage{url}
\usepackage[margin=1in]{geometry}

\hypersetup{
    colorlinks,
    linkcolor={red!50!black},
    citecolor={red!50!black},
    urlcolor={blue!50!black}
}

\newtheorem{theorem}{Theorem}[section]
\newtheorem{definition}{Definition}[section]
\newtheorem{example}{Example}[section]
\newtheorem{corollary}{Corollary}[theorem]
\newtheorem{lemma}[theorem]{Lemma}

\newcommand*{\floor}[1]{\left\lfloor #1 \right\rfloor}
\newcommand*{\ceil}[1]{\left\lceil #1 \right\rceil}

\newcommand{\initFromContents}{
\tableofcontents
\newpage
\setlength{\parindent}{0pt}
\setlength{\parskip}{0.5em}
}

\newcommand{\initAfterBeginDocument}{
\maketitle
\initFromContents{}
}

\newcommand{\addMyBib}{
\bibliographystyle{plainurl}
\bibliography{bibdb}
}

\date{}

\input{src/cmo/common.texlib}

\title{CMO: Constrained optimization for convex functions}

\begin{document}

\maketitle
\initMinimal{}

\section{Convex function and convex constraints}

Let's analyze the following problem:
\[ \begin{array}{lll}
\min_x & f(x) \\
\textrm{where} & c_i(x) \le 0 & \forall i \in I \\
& h_j(x) = 0 & \forall j \in I
\end{array} \]

Here $f$ and $c_i$ are convex and $C^1$ and $h_j$ is linear, i.e. $h_j(x) = a_j^Tx - b_j$.

\subsection{Feasible region is a convex set}

\begin{lemma}[Homework]
The set $S_i = \{x: c_i(x) \le 0\}$ is convex.
\end{lemma}
\begin{lemma}[Homework]
The set $S_j = \{x: h_j(x) = 0\}$ is convex.
\end{lemma}
\begin{lemma}[Homework]
The intersection of convex sets is convex.
\end{lemma}

\subsection{KKT point gives global minimum}

Define the Lagrangian as
\[ L(x, \lambda, \mu) = f(x) + \lambda^Tc(x) + \mu^Th(x) \]

\begin{lemma}
\label{thm:f-ge-lagr}
If $\lambda_i \ge 0$ and $x$ is a feasible point, then $f(x) \ge L(x, \mu, \lambda)$.
\end{lemma}
\begin{proof}
Since $x$ is a feasible point,
\begin{align*}
& c_i(x) \le 0 \wedge h_j(x) = 0
\\ &\implies \lambda^Tc(x) \le 0 \wedge \mu^Th(x) = 0
\\ &\implies f(x) + \lambda^Tc(x) + \mu^Th(x) \le f(x)
\\ &\implies L(x, \lambda, \mu) \le f(x)
\end{align*}
\end{proof}

\begin{lemma}
\label{thm:f-eq-lagr}
Let $(x^*, \lambda^*, \mu^*)$ be a KKT point.
Then $f(x^*) = L(x^*, \mu^*, \lambda^*)$.
\end{lemma}
\begin{proof}
\begin{align*}
& \lambda_i^*c_i(x^*) = 0 \wedge h_j(x^*) = 0 \tag{complementary slackness and primal feasibility}
\\ &\implies {\lambda^*}^Tc(x^*) = 0 \wedge {\mu^*}^Th(x^*) = 0
\\ &\implies f(x^*) + {\lambda^*}^Tc(x^*) + {\mu^*}^Th(x^*) = f(x^*)
\\ &\implies L(x^*, \lambda^*, \mu^*) = f(x)
\end{align*}
\end{proof}

\begin{theorem}[Proved previously]
\label{thm:c1-convex}
Let $f$ be $C^1$ and convex. Then
\[ \forall u, v \in \mathbb{R}^d, f(v) \ge f(u) + \grad_f(u)^T(v-u) \]
\end{theorem}

\begin{theorem}
Let $(x^*, \lambda^*, \mu^*)$ be a KKT point.
Then $x^*$ is a constrained global minimum of $f$.
\end{theorem}
\begin{proof}
Let $x$ be a feasible point.
\begin{align*}
f(x) &\ge L(x, \lambda^*, \mu^*) \tag{by lemma \ref{thm:f-ge-lagr}}
\\ &= f(x) + \sum_i \lambda_i^*c_i(x) + \sum_j \mu_j^*(a_j^Tx - b_j)
\\ &\ge \left(f(x) + \grad_f(x^*)^T(x-x^*)\right)
\\ &\quad + \sum_i \lambda_i^* \left( c_i(x^*) + \grad_{c_i}(x^*)^T(x-x^*) \right)
\\ &\quad + \sum_j \mu_j^* \left( a_j^T(x - x^*) + (a_j^Tx^* - b_j) \right)
\tag{by theorem \ref{thm:c1-convex}}
\\ &= \left( f(x^*) + \sum_i \lambda_i^*c_i(x^*) + \sum_j \mu_j^*(a_j^Tx^* - b_j) \right)
\\ &\quad + (x-x^*)^T\left( \grad_f(x^*) + \sum_i \lambda_i^* \grad_{c_i}(x^*) + \sum_j \mu_j^*a_j \right)
\tag{rearrange terms}
\\ &= L(x^*, \lambda^*, \mu^*) + (x-x^*)^T (\grad_xL)(x^*, \lambda^*, \mu^*)
\\ &= f(x^*) \tag{by lemma \ref{thm:f-eq-lagr} and stationarity}
\end{align*}
Since for all feasible points $f(x) \ge f(x^*)$, $x^*$ is a constrained global minimum of $f$.
\end{proof}

Note that unlike the necessary conditions for local minimum,
here we do not require regularity.

\subsection{Example: Projection over ball}

Consider the optimization problem:
\[ \min_x \frac{1}{2} \norm{x-z}^2 \textrm{ where } \norm{x}^2 \le r^2 \]
Here $z$ lies outside the feasible region.

$\norm{x-z}^2$ and $\norm{x}^2$ are convex functions
(because their hessian is $2I$, which is positive definite),
so this is a convex optimization problem.
\[ L(x, \lambda) = \frac{1}{2}\norm{x-z}^2 + \lambda(\norm{x}^2 - r^2) \]

Applying the KKT conditions, we get
\begin{itemize}
\item Stationarity: $z = (2\lambda + 1)x$.
\item Primal feasibility: $\norm{x}^2 \le r^2$.
\item Dual feasibility: $\lambda \ge 0$.
\item Complementary slackness: $\lambda(\norm{x}^2 - r^2) = 0$.
\end{itemize}

If we take $\lambda = 0$, then stationarity gives us $x = z$.
This violates feasibility, so this is not possible.
Therefore, complementary slackness gives us $\norm{x}^2 = r^2$.
On simplifying, we get
\begin{align*}
x &= \frac{r}{\norm{z}}z
& \lambda &= \frac{1}{2}\left(\frac{\norm{z}}{r}-1\right)
& f(x) &= \frac{1}{2}(r - \norm{z})^2
\end{align*}

\end{document}
