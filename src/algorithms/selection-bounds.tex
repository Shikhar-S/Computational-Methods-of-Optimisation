\documentclass[a4paper, 12pt, fleqn]{article}

\usepackage{amsmath}
\usepackage{amsthm}
\usepackage{amssymb}
\usepackage{xcolor}
\usepackage{float}
\usepackage{hyperref}
\usepackage{url}
\usepackage[margin=1in]{geometry}

\hypersetup{
    colorlinks,
    linkcolor={red!50!black},
    citecolor={red!50!black},
    urlcolor={blue!50!black}
}

\newtheorem{theorem}{Theorem}[section]
\newtheorem{definition}{Definition}[section]
\newtheorem{example}{Example}[section]
\newtheorem{corollary}{Corollary}[theorem]
\newtheorem{lemma}[theorem]{Lemma}

\newcommand*{\floor}[1]{\left\lfloor #1 \right\rfloor}
\newcommand*{\ceil}[1]{\left\lceil #1 \right\rceil}

\newcommand{\initFromContents}{
\tableofcontents
\newpage
\setlength{\parindent}{0pt}
\setlength{\parskip}{0.5em}
}

\newcommand{\initAfterBeginDocument}{
\maketitle
\initFromContents{}
}

\newcommand{\addMyBib}{
\bibliographystyle{plainurl}
\bibliography{bibdb}
}

\date{}


\title{Bounds on Selection}

\begin{document}

\maketitle

\begin{abstract}
This document analyzes lower and upper bounds on the worst-case number of comparisons required for selection,
i.e. finding the $k^{\textrm{th}}$-smallest element in an array of $n$ elements.
\end{abstract}

\initFromContents{}

\section{General lower bound}

There must be at least $n-1$ comparisons [proof needed].

\section{Specific algorithms}

\subsection{Sort and select}

First sort the array, then return the $k^{\textrm{th}}$ element.

Sorting has a lower bound of $\ceil{\lg(n!)}$ comparisons.
It can be upper bounded by considering specific sorting algorithms.
All efficient algorithms use $\Theta(n\lg n)$ comparisons.

\subsection{Build heap and pop}

Build a min-heap of all elements and pop $k$ times.

A binary min-heap of size $n$ can be built using at most $2(n-1)$ comparisons and an item can be
popped from it using at most $2\floor{\lg (n-1)}$ comparisons \cite{eku-notes-heaps}.

Therefore, number of comparisons to select the $k^{\textrm{th}}$-smallest element is upper bounded by
\[ 2\left( n-1 + \sum_{i=1}^k \floor{\lg(n-i)} \right)
\le 2(n-1 + k\floor{\lg(n-1)}) \]

\subsection{Maintain heap and pop}

Build a max-heap of first $k$ elements.
For each of the rest of the elements, compare it with the heap top.
Replace top by new element and repair heap.
The heap top at the end of this process is the $k^{\textrm{th}}$-smallest element.

A binary max-heap of size $k$ can be built using at most $2(k-1)$ comparisons and top can be replaced
using at most $2\floor{\lg k}$ comparisons \cite{eku-notes-heaps}.

Therefore, number of comparisons to select the $k^{\textrm{th}}$-smallest element is upper bounded by
\[ 2((n-k)\floor{\lg k} + (k-1)) \]

Although this is not as efficient as the previous algorithm,
it has the advantage of being an online streaming algorithm.

\subsection{Randomized quick-select}

This algorithms partitions the array using a random pivot
and then recursively selects from one of the partitions.

Let $f(n)$ be the expected number of comparisons required to select from $n$ elements.
$f(0) = f(1) = 0$.

Partitioning the array can be done using $n-1$ comparisons.

\begin{align*}
f(n) &\le (n-1) + \frac{1}{n} \sum_{i=1}^n \max(f(i-1), f(n-i))
\\ &\le (n-1) + \frac{2}{n} \sum_{i=\floor{\frac{n}{2}}}^{n-1} T(i)
\end{align*}

We will prove using mathematical induction that $f(n) \le 4n$.
Since $f(0) = 0$, $d \ge 0$.

As the induction hypothesis, assume $f(i)$ for all $0 \le i \le n-1$.
\begin{align*}
& f(n) \le (n-1) + \frac{2}{n} \sum_{i=\floor{\frac{n}{2}}}^{n-1} T(i)
\\ &\le (n-1) + \frac{2}{n} \sum_{i=\floor{\frac{n}{2}}}^{n-1} 4i
\\ &= (n-1) + \frac{4}{n} \left( n(n-1) -
\floor{\frac{n}{2}}\left(\floor{\frac{n}{2}}-1\right) \right)
\\ &\le (n-1) + \frac{4}{n} \left( n(n-1) -
\frac{n}{2}\left(\frac{n}{2}-1\right) \right)
\\ &= 4n - 3 \le 4n
\end{align*}

Therefore, randomized quick-select performs at most $4n$ expected comparisons
in the worst case.

\subsection{Median-of-medians quick-select}

Median-of-medians quick-select (MoMQS) is a deterministic variant of quick-select which uses
quick-select recursively to decide a good pivot for partitioning.
This guarantees worst-case linear time complexity.

Let $f(n)$ be the worst-case number of comparisons needed to run MoMQS on $n$ elements.
$f(0) = f(1) = 0$.

Steps in MoMQS:
\begin{enumerate}
\item Split array into sub-arrays of length 5 or less such that at most 1 sub-array has length less than 5.
\item Find median of each of those subarrays of length 5.
    It takes at most 6 comparisons \cite{selection-comparisons-table} to find each median.
\item Find the median of these $\ceil{\frac{n}{5}}$ medians. This takes $f(\ceil{\frac{n}{5}})$ comparisons.
\item Partition the array about the median of medians. This takes $n-1$ comparisons.
\item Recurse into the appropriate partition. It can be proven that the larger part of
    the array has size at most around $\frac{7n}{10}$. Therefore at most $\frac{7n}{10}$ comparisons are required.
\end{enumerate}

Therefore, a rough recurrence relation for $f$ is:
\[ f(n) \approx \frac{6n}{5} + f\left(\frac{n}{5}\right) + (n-1) + f\left(\frac{7n}{10}\right) \]

It can be verified that $f(n) \le 22n$.

\addMyBib{}

\end{document}
