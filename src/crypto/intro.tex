\documentclass[a4paper, 12pt, fleqn]{article}

\usepackage{amsmath}
\usepackage{amsthm}
\usepackage{amssymb}
\usepackage{xcolor}
\usepackage{float}
\usepackage{hyperref}
\usepackage{url}
\usepackage[margin=1in]{geometry}

\hypersetup{
    colorlinks,
    linkcolor={red!50!black},
    citecolor={red!50!black},
    urlcolor={blue!50!black}
}

\newtheorem{theorem}{Theorem}[section]
\newtheorem{definition}{Definition}[section]
\newtheorem{example}{Example}[section]
\newtheorem{corollary}{Corollary}[theorem]
\newtheorem{lemma}[theorem]{Lemma}

\newcommand*{\floor}[1]{\left\lfloor #1 \right\rfloor}
\newcommand*{\ceil}[1]{\left\lceil #1 \right\rceil}

\newcommand{\initFromContents}{
\tableofcontents
\newpage
\setlength{\parindent}{0pt}
\setlength{\parskip}{0.5em}
}

\newcommand{\initAfterBeginDocument}{
\maketitle
\initFromContents{}
}

\newcommand{\addMyBib}{
\bibliographystyle{plainurl}
\bibliography{bibdb}
}

\date{}


\title{Introduction to Cryptography}

\begin{document}

\initAfterBeginDocument{}

\textbf{Cryptography}: The study of mathematical techniques for securing
digital information, systems, and distributed computations against adversarial attacks.

Unlike modern cryptography, classical cryptography was based on ad-hoc techniques and lacked rigor.

\section{Private-key encryption}

Alice and Bob want to communicate over a public channel,
but want to keep their communication private from an eavesdropper.
Alice and Bob share a secret key $k$.

A private-key encryption scheme is the tuple
$(\mathcal{M}, \mathcal{K}, \mathcal{C}, \mathsf{Gen}, e, d)$ where:

\begin{itemize}
\item $\mathcal{M}$ is a set called `message space'.
\item $\mathcal{K}$ is a set called `key space'.
\item $\mathcal{C}$ is a set called `cipher space'.
\item The key-generation algorithm $\mathsf{Gen}$ is a probabilistic algorithm
that samples an item from $\mathcal{K}$.
\item The encryption algorithm: For $k \in \mathcal{K}$,
  $e_k: \mathcal{M} \mapsto \mathcal{C}$.
\item The decryption algorithm: For $k \in \mathcal{K}$, $d_k = e_k^{-1}$.
\end{itemize}


\section{Kerckhoffs' Principle}

Kerckhoffs says that the key must be secret, but the encryption scheme should be public.

Disadvantages of requiring the encryption scheme to be secret:

\begin{itemize}
\item Every pair of users will need a new algorithm.
\item If the algorithm is leaked or lost, a new one will have to be invented.
\item An algorithm which hasn't undergone public scrutiny is insecure.
\end{itemize}


\section{Historical ciphers}

All historical ciphers discussed here operate on strings of English characters.
We denote the characters by $\Sigma = \mathbb{Z}_{26}$.
$\mathcal{M} = \mathcal{C} = \Sigma^*$.

\subsection{Caesar Cipher}

\begin{itemize}
\item $\mathcal{K} = \{\}$
\item $e_k(x)[i] = (x[i]+3)\%26$
\item $d_k(x)[i] = (x[i]-3)\%26$
\end{itemize}

Trivial to break, since there is no key.

\subsection{Shift Cipher}

\begin{itemize}
\item $\mathcal{K} = \mathbb{Z}_{26}$; $|\mathcal{K}| = 26$.
\item $e_k(x)[i] = (x[i]+k)\%26$
\item $d_k(x)[i] = (x[i]-k)\%26$
\end{itemize}

Easy to break by brute force since key space is small.
Frequency analysis will make it easier to guess which key to start with.

\subsection{Mono-alphabetic Substitution Cipher}

\begin{itemize}
\item $\mathcal{K} \in S_{\Sigma}$ ($\mathcal{K}$ is a permutation of $\Sigma$);
  $|\mathcal{K}| = 26! \approx 4.03 \times 10^{26}$.
\item $e_k(x)[i] = k(x[i])$
\item $e_d(x)[i] = k^{-1}(x[i])$
\end{itemize}

Can be broken using frequency analysis of the message space.

\subsection{Vigen\`ere Cipher}

Also known as Poly-alphabetic shift cipher.

\begin{itemize}
\item $\mathcal{K} = \Sigma^*$.
\item $e_k(x)[i] = (x[i] + k[i\%|k|])\%26$.
\item $d_k(x)[i] = (x[i] - k[i\%|k|])\%26$.
\end{itemize}

If the key length is known, it can be broken using frequency analysis of every stream.
For the cipher-text $c$ and key-length $l$, the $i^{\textrm{th}}$ stream is the sequence
$\texttt{c[i::l]}$ (python slice notation).

The key length can also be guessed using frequency analysis,
like Kasisiki's method or using mean-square-frequency.

\section{Definition of security}

To mathematically prove that a cryptographic protocol is secure,
we have to formally define what we mean by security.

There can be multiple definitions of security depending on the application and environment.
Before developing a cryptographic solution to a problem, we must
choose the definition of security that is most relevant to the application.

A security definition has 2 components: a security guarantee and a threat model.

\textbf{Security guarantee for secure communication}:
Regardless of any information an attacker already has,
a ciphertext should leak no additional information about the underlying plaintext.

\subsection{Standard threat models for secure communication}

\begin{itemize}
\item \textbf{Ciphertext-only attack}:
Given $S_C = \{e_k(m): m \in S_M\}$ find out something about $S_M$.

\item \textbf{Known-plaintext attack}:
Given $T = \{(m, e_k(m)): m \in S\}$ and $S_C = \{e_k(m): m \in S_M\}$ find out something about $S_M$.

\item \textbf{Chosen-plaintext attack}:
Given $S_C = \{e_k(m): m \in S_M\}$ and black-box access to $e_k$,
find out something about $S_M$.

\item \textbf{Chosen-ciphertext attack}:
Given $S_C = \{e_k(m): m \in S_M\}$ and black-box access to $e_k$ and $d_k$,
find out something about $S_M$. The attacker is not allowed to feed elements of $S_C$ to $d_k$.
\end{itemize}

\end{document}
